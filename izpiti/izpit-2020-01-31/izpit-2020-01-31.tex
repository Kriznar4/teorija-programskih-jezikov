\documentclass[arhiv]{../izpit}
\usepackage{amssymb}
\usepackage{fouriernc}

\begin{document}

\newcommand{\bnfis}{\mathrel{{:}{:}{=}}}
\newcommand{\bnfor}{\;\mid\;}
\newcommand{\fun}[2]{\lambda #1. #2}
\newcommand{\conditional}[3]{\mathtt{if}\;#1\;\mathtt{then}\;#2\;\mathtt{else}\;#3}
\newcommand{\whileloop}[2]{\mathtt{while}\;#1\;\mathtt{do}\;#2}
\newcommand{\recfun}[3]{\mathtt{rec}\;\mathtt{fun}\;#1\;#2. #3}
\newcommand{\boolty}{\mathtt{bool}}
\newcommand{\funty}[2]{#1 \to #2}
\newcommand{\tru}{\mathtt{true}}
\newcommand{\fls}{\mathtt{false}}
\newcommand{\tbool}{\mathtt{bool}}
\newcommand{\tand}{\mathbin{\mathtt{and}}}
\newcommand{\tandalso}{\mathbin{\mathtt{andalso}}}
\newcommand{\imp}{\textsc{imp}}
\newcommand{\skp}{\mathtt{skip}}
\newcommand{\nadaljevanje}{\dodatek{\newpage\noindent\emph{(nadaljevanje rešitve \arabic{naloga}. naloge)}}}
\izpit
  [ucilnica=P02]{Teorija programskih jezikov: 1. izpit}{31.\ januar 2020}{
}
\dodatek{
  \vspace{\stretch{1}}
  \begin{itemize}
    \item \textbf{Ne odpirajte} te pole, dokler ne dobite dovoljenja.
    \item Zgoraj \textbf{vpišite svoje podatke} in označite \textbf{sedež}.
    \item Na vidno mesto položite \textbf{dokument s sliko}.
    \item Preverite, da imate \textbf{telefon izklopljen} in spravljen.
    \item Čas pisanja je \textbf{180 minut}.
    \item Doseči je možno \textbf{80 točk}.
    \item Veliko uspeha!
  \end{itemize}
  \vspace{\stretch{3}}
  \newpage
}

%%%%%%%%%%%%%%%%%%%%%%%%%%%%%%%%%%%%%%%%%%%%%%%%%%%%%%%%%%%%%%%%%%%%%%%

\naloga[\tocke{20}]
V $\lambda$-računu lahko vsako naravno število $n$ predstavimo s \emph{Churchevim naravnim številom}
\[
  e_n = \fun{s} \fun{z} \underbrace{s (s (\cdots (s}_n \; z) \cdots))
\]
\podnaloga Zapišite vse korake v evalvaciji izraza $(e_2 \; (\fun{n} 2 * n)) \; 1$.
\podnaloga Napišite funkcijo $f$, ki vsako Churchevo naravno število pretvori v običajno naravno število. Veljati mora torej $f \; e_n \leadsto^* n$.
\podnaloga Izračunajte najbolj splošen tip izraza $e_2$.
\nadaljevanje
%%%%%%%%%%%%%%%%%%%%%%%%%%%%%%%%%%%%%%%%%%%%%%%%%%%%%%%%%%%%%%%%%%%%%%%

\naloga[\tocke{20}]
Operacijsko semantiko programskega jezika {\imp} z ukazi
\[
  c \bnfis
  \conditional{b}{c_1}{c_2} \bnfor
  \whileloop{b}{c} \bnfor
  c_1; c_2 \bnfor
  \ell := e \bnfor
  \skp
\]
smo običajno podali z velikimi koraki, torej z relacijami
\[
  s, e \Downarrow n, \qquad s, b \Downarrow p, \qquad s, c \Downarrow s',
\]
vendar bi lahko podobno storili tudi z malimi koraki
\[
  s, e \leadsto e', \qquad s, b \leadsto b', \qquad s, c \leadsto s', c'
\]
Pri tem aritmetični izrazi s koraki končajo, ko dosežejo število $n$, Booleovi izrazi takrat, ko dosežejo logično vrednost $p$, ukazi pa takrat, ko dosežejo $s, \skp$.

\podnaloga Podajte pravila, ki določajo relacijo $s, c \leadsto s', c'$. Pravil za aritmetične in Booleove izraze vam ni treba podajati.

\newcommand{\interleave}[2]{#1 \leftrightharpoons #2}
\podnaloga Jezik {\imp} razširimo s prepletenim izvajanjem $\interleave{c_1}{c_2}$, ki najprej izvede en korak ukaza $c_1$, nato en korak ukaza $c_2$, nato naslednji korak ukaza $c_1$ in tako naprej. Če je ukaz, ki je na vrsti, z izvajanjem končal, do konca izvedemo preostali ukaz. Zapišite dodatna pravila, s katerimi morate razširiti operacijsko semantiko.
\prostor[2]

\nadaljevanje
%%%%%%%%%%%%%%%%%%%%%%%%%%%%%%%%%%%%%%%%%%%%%%%%%%%%%%%%%%%%%%%%%%%%%%%

\naloga[\tocke{20}]
\newcommand{\fix}{\mathit{fix}}
Dana naj bo domena $D$ in naj bo $\fix \colon [D \to D] \to D$ preslikava, ki vsaki zvezni preslikavi $f \colon D \to D$ priredi njeno najmanjšo fiksno točko $\fix(f) \in D$. Dokažite, da je preslikava $\fix$ zvezna.

\nadaljevanje
%%%%%%%%%%%%%%%%%%%%%%%%%%%%%%%%%%%%%%%%%%%%%%%%%%%%%%%%%%%%%%%%%%%%%%%

\naloga[\tocke{20}]
\newcommand{\good}[1]{\mathcal{D}_{#1}}
Imejmo $\lambda$-račun samo z Booleovimi vrednostmi in funkcijami
\[
  e \bnfis
  x \bnfor
  \tru \bnfor
  \fls \bnfor
  \conditional{e}{e_1}{e_2} \bnfor
  \fun{x}{e} \bnfor
  e_1 \; e_2
\]
ter z običajnimi pravili za neučakano operacijsko semantiko malih korakov in določanje tipov. Ker v jeziku nimamo rekurzije, vsak program, ki ima tip, tudi konvergira, vendar preprosta indukcija žal ne zadošča za dokaz.\footnote{Za čast in slavo lahko poiščete izraz $e$, ki divergira, torej da obstaja neskončno zaporedje
$e \leadsto e_1 \leadsto e_2 \leadsto \cdots$
}


Namesto tega za vsak tip $A$ definiramo množico \emph{dobrih izrazov}~$\good{A}$ kot
%
\begin{align*}
  \good{\boolty} &=
    \big\{ e \mid (e \leadsto^* \tru) \lor (e \leadsto^* \fls) \big\} &
  \good{\funty{A_1}{A_2}} &=
    \big\{ e \mid \forall e' \in \good{A_1}. e \; e' \in \good{A_2} \big\}
\end{align*}
%
in dokaz razbijemo na tri dele\ldots

\podnaloga
Dokažite, da vsi dobri izrazi konvergirajo, torej da za vsak tip $A$ in vsak $e \in \good{A}$ obstaja vrednost~$v$, da velja $e \leadsto^* v$.

\podnaloga
Dokažite, da za vsak tip $A$ iz $e \leadsto e'$ in $e' \in \good{A}$ sledi $e \in \good{A}$.

\podnaloga
Dokažite, da za vsak izraz $x_1 : A_1, \dots, x_n : A_n \vdash e : A$ in vse vrednosti $v_1 \in \good{A_1}, \dots, v_n \in \good{A_n}$ velja tudi $e[v_1 / x_1, \dots, v_n / x_n] \in \good{A}$. Torej: vsak izraz, ki ima tip, je dober, če vse proste spremenljivke zamenjamo z dobrimi vrednostmi.

\nadaljevanje


\izpit
  [ucilnica=P02,anglescina]{Theory of programming languages: first exam}{31 January 2020}{
}
\dodatek{
  \vspace{\stretch{1}}
  \begin{itemize}
    \item \textbf{Do not open} the exam until you are allowed to.
    \item Enter \textbf{your data} and mark your \textbf{seat}.
    \item Ensure that you \textbf{display your ID}.
    \item Ensure that your \textbf{phone is turned off} and stowed away.
    \item You have \textbf{180 minutes}.
    \item You can achieve \textbf{80 marks}.
    \item Good luck!
  \end{itemize}
  \vspace{\stretch{3}}
  \newpage
}

%%%%%%%%%%%%%%%%%%%%%%%%%%%%%%%%%%%%%%%%%%%%%%%%%%%%%%%%%%%%%%%%%%%%%%%

\naloga[\tocke{20}]
In $\lambda$-calculus, each natural number $n$ can be represented by a \emph{Church natural number}
\[
  e_n = \fun{s} \fun{z} \underbrace{s (s (\cdots (s}_n \; z) \cdots))
\]
\podnaloga Write down all the steps in the evaluation of $(e_2 \; (\fun{n} 2 * n)) \; 1$.
\podnaloga Write a function $f$, which converts every Church natural number to an ordinary natural number, ie. $f \; e_n \leadsto^* n$ must hold.
\podnaloga Compute the most general type of the expression $e_2$.
\nadaljevanje
%%%%%%%%%%%%%%%%%%%%%%%%%%%%%%%%%%%%%%%%%%%%%%%%%%%%%%%%%%%%%%%%%%%%%%%

\naloga[\tocke{20}]
Operational semantics of the {\imp} language with commands
\[
  c \bnfis
  \conditional{b}{c_1}{c_2} \bnfor
  \whileloop{b}{c} \bnfor
  c_1; c_2 \bnfor
  \ell := e \bnfor
  \skp
\]
is usually given with big steps, ie. with relations
\[
  s, e \Downarrow n, \qquad s, b \Downarrow p, \qquad s, c \Downarrow s',
\]
though we could do the same with small steps
\[
  s, e \leadsto e', \qquad s, b \leadsto b', \qquad s, c \leadsto s', c'
\]
In this case, arithmetic expressions end evaluation when they reach a number $n$, Boolean expressions when they reach a truth value $p$, and the commands when they reach $s, \skp$.

\podnaloga Give the rules that determine the relation $s, c \leadsto s', c'$. You do not need to give the rules for arithmetic and Boolean expressions.

\podnaloga We extend {\imp} with interleaving $\interleave{c_1}{c_2}$, which first executes one step of $c_1$, then one step of $c_2$, then the next step of $c_1$ and so on. If the command in turn has finished the execution, we evaluate the remaining command. Write the additional rules with which you need to extend the operational semantics.
\prostor[2]

\nadaljevanje
%%%%%%%%%%%%%%%%%%%%%%%%%%%%%%%%%%%%%%%%%%%%%%%%%%%%%%%%%%%%%%%%%%%%%%%

\naloga[\tocke{20}]
Let $D$ be a domain and let $\fix \colon [D \to D] \to D$ map each continuous function $f \colon D \to D$ into its least fixed point $\fix(f) \in D$. Prove that $\fix$ is continuous.

\nadaljevanje
%%%%%%%%%%%%%%%%%%%%%%%%%%%%%%%%%%%%%%%%%%%%%%%%%%%%%%%%%%%%%%%%%%%%%%%

\naloga[\tocke{20}]
Take $\lambda$-calculus with booleans and functions
\[
  e \bnfis
  x \bnfor
  \tru \bnfor
  \fls \bnfor
  \conditional{e}{e_1}{e_2} \bnfor
  \fun{x}{e} \bnfor
  e_1 \; e_2
\]
and usual rules for eager small-step operational semantics and typing judgements. Since the language does not feature recursion, each well-typed expression converges, though simple induction is not sufficient for the proof,.\footnote{For honor and glory you can find an expression $e$, which diverges, ie. there exists an infinite sequence $e \leadsto e_1 \leadsto e_2 \leadsto \cdots$
}


Instead, we define a set of \emph{good expressions}~$\good{A}$ for each type $A$ as
%
\begin{align*}
  \good{\boolty} &=
    \big\{ e \mid (e \leadsto^* \tru) \lor (e \leadsto^* \fls) \big\} &
  \good{\funty{A_1}{A_2}} &=
    \big\{ e \mid \forall e' \in \good{A_1}. e \; e' \in \good{A_2} \big\}
\end{align*}
%
and split the proof into three parts\ldots

\podnaloga
Prove that all good expressions converge, ie. for each type $A$ and each $e \in \good{A}$ there exists a value~$v$, such that $e \leadsto^* v$.

\podnaloga
Prove that $e \leadsto e'$ and $e' \in \good{A}$ imply $e \in \good{A}$ for each type $A$.

\podnaloga
Prove that for each expression $x_1 : A_1, \dots, x_n : A_n \vdash e : A$ and all values $v_1 \in \good{A_1}, \dots, v_n \in \good{A_n}$ we have $e[v_1 / x_1, \dots, v_n / x_n] \in \good{A}$. Ie. each well-typed expression is good, if we replace all free variables with good values.

\nadaljevanje

\end{document}
