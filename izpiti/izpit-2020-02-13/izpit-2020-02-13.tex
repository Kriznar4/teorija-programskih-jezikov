\documentclass[arhiv]{izpit}
\usepackage{amssymb}
\usepackage{fouriernc}

\begin{document}

\newcommand{\bnfis}{\mathrel{{:}{:}{=}}}
\newcommand{\bnfor}{\;\mid\;}
\newcommand{\fun}[2]{\lambda #1. #2}
\newcommand{\conditional}[3]{\mathtt{if}\;#1\;\mathtt{then}\;#2\;\mathtt{else}\;#3}
\newcommand{\whileloop}[2]{\mathtt{while}\;#1\;\mathtt{do}\;#2}
\newcommand{\recfun}[3]{\mathtt{rec}\;\mathtt{fun}\;#1\;#2. #3}
\newcommand{\boolty}{\mathtt{bool}}
\newcommand{\intty}{\mathtt{int}}
\newcommand{\funty}[2]{#1 \to #2}
\newcommand{\tru}{\mathtt{true}}
\newcommand{\fls}{\mathtt{false}}
\newcommand{\tbool}{\mathtt{bool}}
\newcommand{\tand}{\mathbin{\mathtt{and}}}
\newcommand{\tandalso}{\mathbin{\mathtt{andalso}}}
\newcommand{\imp}{\textsc{imp}}
\newcommand{\skp}{\mathtt{skip}}
\newcommand{\nadaljevanje}{\dodatek{\newpage\noindent\emph{(nadaljevanje rešitve \arabic{naloga}. naloge)}}}
\izpit
  [ucilnica=P02,naloge=-1]{Teorija programskih jezikov: 2. izpit}{13.\ februar 2020}{
}
\dodatek{
  \vspace{\stretch{1}}
  \begin{itemize}
    \item \textbf{Ne odpirajte} te pole, dokler ne dobite dovoljenja.
    \item Zgoraj \textbf{vpišite svoje podatke} in označite \textbf{sedež}.
    \item Na vidno mesto položite \textbf{dokument s sliko}.
    \item Preverite, da imate \textbf{telefon izklopljen} in spravljen.
    \item Čas pisanja je \textbf{180 minut}.
    \item Doseči je možno \textbf{80 točk}.
    \item Veliko uspeha!
  \end{itemize}
  \vspace{\stretch{3}}
  \newpage
}

%%%%%%%%%%%%%%%%%%%%%%%%%%%%%%%%%%%%%%%%%%%%%%%%%%%%%%%%%%%%%%%%%%%%%%%

\naloga[\tocke{20}]
V $\lambda$-računu definirajmo funkcijo višjega reda, ki predstavlja komponiranje funkcij:
\[
  \mathit{cmp} = \fun{f}{\Big(\fun{g}{\big(\fun{x}{f (g x)}\big)}\Big)}
\]
\podnaloga Zapišite drevo izpeljave za $\Big(\big(\mathit{cmp} \, (\fun{m} 7 * m)\big) \, (\fun{n} 3 * n)\Big) \, 2 \Downarrow 42$.
\podnaloga Zapišite vse korake v evalvaciji izraza $\Big(\big(\mathit{cmp} \, (\fun{m} 7 * m)\big) \, (\fun{n} 3 * n)\Big) \, 2$ v semantiki malih korakov.
\podnaloga Izračunajte najbolj splošen tip izraza $\mathit{cmp}$.
\nadaljevanje
%%%%%%%%%%%%%%%%%%%%%%%%%%%%%%%%%%%%%%%%%%%%%%%%%%%%%%%%%%%%%%%%%%%%%%%

\naloga[\tocke{20}]

\newcommand{\printint}[2]{\mathtt{print}_{#1}(#2)}
V $\lambda$-račun dodamo izraz za izpisovanje celih števil:
\[
  e \bnfis
  x \bnfor
  \tru \bnfor
  \fls \bnfor
  \conditional{e}{e_1}{e_2} \bnfor
  \fun{x}{e} \bnfor
  e_1 \; e_2 \bnfor
  \printint{e}{e'}
\]
Izraz $\printint{e}{e'}$ izpiše celo število, predstavljeno z izrazom $e$, in nadaljuje z izvajanjem izraza $e'$. Na primer, izraz $e_{42} = \printint{2 + 2}{\printint{1 + 1}{\tru}}$ najprej izpiše $4$, nato $2$ in vrne vrednost $\tru$.

\podnaloga
  Zapišite pravilo za določanje tipa izraza $\printint{e}{e'}$.

\podnaloga
  Za razširjeni $\lambda$-račun zapišite pravila za semantiko velikih korakov oblike $e \Downarrow v, [n_1, \dots, n_k]$, ki poleg končne vrednosti $v$ vrne tudi seznam izpisanih števil. Na primer, zgornji izraz bi se evalviral kot $e_{42} \Downarrow \tru, [4, 2]$.

\nadaljevanje
%%%%%%%%%%%%%%%%%%%%%%%%%%%%%%%%%%%%%%%%%%%%%%%%%%%%%%%%%%%%%%%%%%%%%%%

\naloga[\tocke{20}]
\newcommand{\error}{\mathtt{err}}
\newcommand{\handle}[2]{\mathtt{handle}\;#1\;\mathtt{with}\;#2}
V $\lambda$-račun dodamo eno izjemo in njeno prestrezanje s sledečo sintakso:
\begin{align*}
  e &\bnfis
  x \bnfor
  \tru \bnfor
  \fls \bnfor
  \conditional{e}{e_1}{e_2} \bnfor
  \fun{x}{e} \bnfor
  e_1 \; e_2 \bnfor
  \error \bnfor
  \handle{e_1}{e_2} \\
  v &\bnfis
  \tru \bnfor
  \fls \bnfor
  \fun{x}{e} \bnfor
  \error
\end{align*}
s sledečima dodatnima praviloma za določanje tipov:
\[
\frac{
  \hbox{}
}{
  \Gamma \vdash \error : A
}
\qquad
\frac{
  \Gamma \vdash e_1 : A
  \qquad
  \Gamma \vdash e_2 : A
}{
  \Gamma \vdash \handle{e_1}{e_2} : A
}
\]
ter s sledečimi dodatnimi pravili za operacijsko semantiko:
\[
\frac{
  \hbox{}
}{
  \conditional{\error}{e_1}{e_2} \leadsto \error
}
\qquad
\frac{
  \hbox{}
}{
  \error \, e_2 \leadsto \error
}
\qquad
\frac{
  \hbox{}
}{
  v_1 \, \error \leadsto \error
}
\]
\[
\frac{
  e_1 \leadsto e_1'
}{
  \handle{e_1}{e_2} \leadsto \handle{e_1'}{e_2}
}
\qquad
\frac{
}{
  \handle{\error}{e_2} \leadsto e_2
}
\qquad
\frac{
  v_1 \neq \error
}{
  \handle{v_1}{e_2} \leadsto v_1
}
\]
Dokažite izrek o varnosti za zgornjo razširitev. Pri tem je dovolj, da v dokazu naštejete le spremembe glede na dokaz s predavanj.

\nadaljevanje
%%%%%%%%%%%%%%%%%%%%%%%%%%%%%%%%%%%%%%%%%%%%%%%%%%%%%%%%%%%%%%%%%%%%%%%

\naloga[\tocke{20}]
\newcommand{\sub}{\sqsubseteq}
\newcommand{\pow}[1]{\mathcal{P}(#1)}
\newcommand{\powd}[1]{\mathcal{P}_{\downarrow}(#1)}
\newcommand{\down}[1]{\mathop{\downarrow} #1}
Za denotacijsko semantiko nedeterminističnih programov uporabljamo potenčne množice. Za domeno $(D, \leq)$ in množici $X, Y \subseteq D$ definiramo
\[
  X \sub Y \iff \forall x \in X. \exists y \in Y. x \leq y
  \qquad\text{in}\qquad
  \down{X} = \{ x' \in D \mid \exists x \in X. x' \leq x \}
\]
ter za množico $X$ pravimo, da je \emph{navzdol zaprta}, če velja $X = \down{X}$. Naj bo $\powd{D}$ množica vseh podmnožic $D$, ki so (1) neprazne, (2) navzdol zaprte in (3) zaprte za supremume naraščajočih verig.

\podnaloga Dokažite, da je $(\powd{D}, \sub)$ domena. Namig: supremum verige $X_1 \sub X_2 \sub \cdots$ je enak $\overline{\bigcup_i X_i}$ (to morate dokazati), kjer $\overline{X}$ označuje množico $X$, razširjeno s supremumi vseh naraščajočih verig v~$X$.
\podnaloga Dokažite, da je preslikava $\eta : D \to \powd{D}$, ki vsakemu elementu $x \in D$ priredi $\down{\{ x \}}$, dobro definirana in zvezna.

\nadaljevanje

\end{document}